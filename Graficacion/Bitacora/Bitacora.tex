\documentclass[10pt,a4paper]{article}
\usepackage[latin1]{inputenc}
\usepackage{amsmath}
\usepackage{amsfonts}
\usepackage{amssymb}
\usepackage{graphicx}
\usepackage{multicol}
\usepackage{changepage}
\usepackage{float}
\usepackage{cite}
\usepackage{url}
\usepackage{imakeidx}
\makeindex

\usepackage[left=2.50cm, right=2.50cm]{geometry}
\usepackage[spanish]{babel}

\author{Nombre}
\title{Portada siempre practica}

\begin{document}
%encabezado 
\pagestyle{plain}{
\pagestyle{empty}
\changepage{3cm}{1cm}{-0.5cm}{-0.5cm}{}{-2cm}{}{}{}
\noindent

%sEGIUN EL formato de sus imagenes, deben encontrar una configuracion adeacuada para ustedes
{\small
\begin{tabular}{p{0.626\textwidth} p{0.50\textwidth} }
\includegraphics[scale=0.26]{uaem.jpg} &  \includegraphics[scale=0.3]{ico.jpg}
\end{tabular}
}

%datos de la caratula
\begin{center}
\par\vspace{2cm} %Rspacoo dejado antes del encabezado
{
\Huge\textbf{
Universidad Aut\'onoma del Estado de M\'exico \\[1cm] Ingenieria en Computaci\'on
}
}
\par\vspace{1.5cm}
{
\Large\textbf{ Materia: Graficaci\'on
}
}
\par\vspace{1.5cm}
{
\large\textbf{Axel Valenzuela Ju\'arez \\ 31 de Mayo del 2019  } 
}
\par\vspace{1.5cm}

\end{center}
\clearpage

}

\printindex

\tableofcontents % indice de contenidos

\cleardoublepage
\addcontentsline{toc}{section}{Lista de figuras} % para que aparezca en el indice de contenidos
\listoffigures % indice de figuras

\cleardoublepage

\section{Clase de 15 de febrero\index{Clase}
}

\paragraph{
Se defini\'o la escala as\'i como los criterios a evaluar, el profesor nos mostr\'o la herramienta a utilizar todo el semestre y nos ayud\'o a instalar octave, una vez pasamos por todo el proceso de descarga y descargamos octave desde la p\'agina oficial.
}
\paragraph{
Una vez instalado octave el profesor procedi\'o a explicarnos la sintaxis de Octave as\'i como la manera de compilar y revisar los errores.
}

\paragraph{Cada imagen esta compuesta de un numero finito de elementos, son referidos como elementos de imagen o pixel.}

\section{
Clase de 22 de febrero
}
\paragraph{
Binarizaci\'on: es el proceso de volver todos los n\'umeros a unos y ceros esto en una imagen provoca pasarla a blancos y negros puros, realizamos en octave el proceso de binarizaci\'on , esto se realiz\'o con dos ciclos for y poniendo como condiciones que si el color era mayor que 126 se pasaba a negro y si era menor a 126 se pasaba a blancos de esa manera a la imagen se le borraron los grises claros y oscuros.
}

\paragraph{
Despu\'es de realizar la binarizacion experimentamos con distintas gr\'aficas y a las im\'agenes les dimos distintos tipos de grises de esa manera empezamos a aplicar filtros a las im\'agenes los resultados pueden ser observados en las imagenes \ref{fig:Binarizacion} ,\ref{fig:Binarizacion2},\ref{fig:Binarizacion3}.
}
\begin{figure}[h]
\includegraphics[scale=0.6] {Binarizacion.png}
\caption{Imagen binarizada a negros y blancos puros}
\label{fig:Binarizacion}
\end{figure}

\section{Referencias}
\paragraph{$http://minisconlatex.blogspot.com/2011/04/como-escribir-una-tesis-con-latex.html$ Recuperado el 28 de mayo del 2019}
\paragraph{$Digital Image Processing, Rafael C. Gonzalez,1977$ Recuperado el 1 de mayo del 2019}


\end{document}